\#\+Symulator Pająka\+:\hypertarget{md_README_Podstawowe}{}\section{zalozenia\+:}\label{md_README_Podstawowe}

\begin{DoxyItemize}
\item Na początku tworzony jest obiekt pająk, w klasie Pająk-\/ posiadać on będzie atrybuty tj punkty akcji, hp wielkość , siła ataku, współczynnik regeneracji oraz metody\+: atak, atak specjalny (trucizna),rage, regeneracja, budowanie sieci, odżywianie, ucieczka.
\item Dodatkowo będzie istnieć klasa Pajęczyna o atrybutach\+: wielkość, wytrzymałość oraz metodzie\+: zerwanie.
\item Na początku symulacji pająk będzie zaczynał na pajęczynie o określonej, $<$ma łej$>$=\char`\"{}\char`\"{}$>$ wielkości. W każdej turze na pajęczyne losowo łapać się będą muchy oraz inne większe owady np pszczoła.
\item Spożywanie swoich zdobyczy będzie zwiększało statystyki naszego pająka.
\item Większy rozmiar tuptusia oznaczać będzie konieczność rozbudowy pajęczyny. Po przekroczeniu pewnej wielkości sieci-\/ stanie się ona widzialna dla innych pająków. Co zainizjuje walkę o pajęczyne.
\end{DoxyItemize}\hypertarget{md_README_Dodatkowe}{}\section{informacje\+:}\label{md_README_Dodatkowe}

\begin{DoxyItemize}
\item Symulator będzie turówką, pająk posiada punkty akcji.
\item Decyzja na co zostaną rozdzielone PA, zapadac będzie w oparciu o wartości\+: zdrowie(mało hp idź zjedz muche), stan pajęczyny(sieć się zarwie-\/ idź ją napraw) -\/ obliczane wg odpowiedniego algorytmu.
\item Złapane owady, jeżeli nie zostaną wykorzystane w przeciągu 1-\/3 tur odpadają z pajęczyny i przepadają.
\item Pająk, który nie jadł przez 2 (?) Tury ginie.
\item Sieć może być niszczona\+: w czasie walki, w każdej turze od warunków zewnętrznych np deszcz, wiatr.
\item Nie naprawiana sieć ulega zniszczeniu.
\item W czasie walki i małej ilości hp pająk może uciec z pajęczyny i założyć nowa. 
\end{DoxyItemize}